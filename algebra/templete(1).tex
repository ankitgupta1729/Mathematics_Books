\documentclass[11pt]{article}
\usepackage{amsfonts}
\usepackage{latexsym}
\usepackage{graphicx}
\setlength{\oddsidemargin}{.25in}
\setlength{\evensidemargin}{.25in}
\setlength{\textwidth}{6in}
\setlength{\topmargin}{-0.4in}
\setlength{\textheight}{8.5in}

\newtheorem{theorem}{Theorem}[section]
\newtheorem{fact}[theorem]{Fact}
\newtheorem{proposition}[theorem]{Proposition}
\newtheorem{observation}[theorem]{Observation}
\newtheorem{lemma}[theorem]{Lemma}
\newtheorem{definition}[theorem]{Definition}
\newtheorem{corollary}[theorem]{Corollary}
\newtheorem{remark}[theorem]{Remark}
\newtheorem{claim}[theorem]{Claim}
\newtheorem{conjecture}[theorem]{Conjecture}
\newtheorem{assumption}[theorem]{Assumption}

\newcommand{\qed}{\hfill \ensuremath{\Box}}

\newenvironment{proof}{
\vspace*{-\parskip}\noindent\textit{Proof.}}{$\qed$

\medskip
}

\newcommand{\alg}[1]{\mathsf{#1}}
\newcommand{\Prover}{\alg{P}}
\newcommand{\Verifier}{\alg{V}}
\newcommand{\Simulator}{\alg{S}}
\newcommand{\PPT}{\alg{PPT}}
\newcommand{\isom}{\cong}
\newcommand{\from}{\stackrel{\scriptstyle R}{\leftarrow}}
\newcommand{\handout}[5]{
   \renewcommand{\thepage}{#1-\arabic{page}}
   \noindent
   \begin{center}
   \framebox{
      \vbox{
    \hbox to 5.78in { {\bf Discrete Mathematics} \hfill #2 }
       \vspace{4mm}
       \hbox to 5.78in { {\Large \hfill #5  \hfill} }
       \vspace{2mm}
       \hbox to 5.78in { {\it #3 \hfill #4} }
      }
   }
   \end{center}
   \vspace*{4mm}
}

\newcommand{\ho}[5]{\handout{#1}{#2}{Instructor:
#3}{Scribe: #4}{Lecture #1: #5}}
\newcommand{\al}{\alpha}
\newcommand{\Z}{\mathbb Z}

\begin{document}
\ho{1}{DATE}{Sourav Chakraborty} {YOUR NAME HERE} {TITLE}

%
% NOTE: If you want to scan in hand-drawings for figures, that is fine.
%

\section{First Section}

\subsection{Topics}

\begin{itemize}
\item Topic 1
\item Topic 2
\end{itemize}

\begin{enumerate}
\item Topic 3
\item Topic 4
\end{enumerate}

\subsection{Proofs}

\begin{theorem}[Discrete Mathematics]
\label{thm:one}
There exists a person who likes discrete mathematics.
\end{theorem}

\begin{proof}
I do!
\end{proof}

\subsection{Figures}
\label{sec:figs}

 \begin{figure}[h]
   \centering
   % uncomment out line below
   \includegraphics[width=2in]{figure.png}
   \caption{Write caption here.}
   \label{fig:sample}
 \end{figure}

\subsection{References}

This is how you reference Section~\ref{sec:figs}, Theorem~\ref{thm:one} or Figure~\ref{fig:sample}.   You can also make a citation~\cite{XYZ14}.

\bibliography{bibfile}
\bibliographystyle{plain}

\end{document}
